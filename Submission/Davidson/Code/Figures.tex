\documentclass{article}
\usepackage[utf8]{inputenc}

% use the "wcp" class option for workshop and conference
 % proceedings
 %\documentclass[gray]{jmlr} % test grayscale version
 %\documentclass[tablecaption=bottom]{jmlr}% journal article
 %\pdfoutput=1
 \documentclass[pmlr,twocolumn,10pt]{jmlr} % W&CP article
 
 \let\SUP\textsuperscript

% \usepackage{geometry}
% \geometry{margins=0.1in,textwidth=7in}

 % The following packages will be automatically loaded:
 % amsmath, amssymb, natbib, graphicx, url, algorithm2e

 %\usepackage{rotating}% for sideways figures and tables
 %\usepackage{longtable}% for long tables

 % The booktabs package is used by this sample document
 % (it provides \toprule, \midrule and \bottomrule).
 % Remove the next line if you don't require it.
\usepackage{tabto}    
\usepackage{bm}
\usepackage{graphicx}
\graphicspath{ {images/} }
\usepackage{float}
\usepackage{natbib}
\usepackage{url}
\usepackage{textcomp}
% \usepackage{syntonly}
% \syntaxonly

\usepackage{booktabs}
 % The siunitx package is used by this sample document
 % to align numbers in a column by their decimal point.
 % Remove the next line if you don't require it.    
\usepackage[load-configurations=version-1]{siunitx} % newer version

%\usepackage{siunitx}

 % The following command is just for this sample document:
\newcommand{\cs}[1]{\texttt{\char`\\#1}}% remove this in your real article

% The following is to recognise equal contribution for authorship
\newcommand{\equal}[1]{{\hypersetup{linkcolor=black}\thanks{#1}}}

% Customized Tabs
\newcommand\mytab{\hspace{10mm} \hspace{-5cm}}



\begin{document}

% Appendix D
\begin{table}[hbtp] 
\floatconts

\centering
    \begin{tabular}{lcc}
        \toprule
        \bfseries Keyword & \bfseries Match Count \\
        \midrule
        
        \bfseries Memory & 109218 \\ 
        \bfseries Cognition & 87655 \\ 
        \bfseries Dementia & 51034 \\ 
        \bfseries Cerebral & 45886 \\ 
        \bfseries Cerebrovascular & 36370 \\ 
        \bfseries Cerebellar & 26863 \\
        \bfseries Cognitive Impairment & 20267 \\ 
        \bfseries Alzheimer & 20581 \\ 
        \bfseries MOCA & 9767 \\ 
        \bfseries Neurocognitive & 7711 \\ 
        \bfseries MCI & 3889 \\ 
        \bfseries Amnesia & 3695 \\ 
        \bfseries AD & 2673 \\ 
        \bfseries Lewy & 2561 \\ 
        \bfseries MMSE & 2134 \\ 
        \bfseries LBD & 224 \\ 
        \bfseries Corticobasal & 147 \\ 
        \bfseries Picks & 41 \\ 
        
        \bottomrule
        \end{tabular}
        {\caption{Keywords indicative of Cognitive Impairment}}
\end{table}

\begin{table}[htb]
\floatconts
\centering
\resizebox{\columnwidth}{!}{
        \begin{tabular}{lccc}
            \toprule
            \bfseries Characteristic & \bfseries (N = 16428) \\
            \midrule
            \bfseries Age (years) mean (SD) & 73.01 (7.96) \\ 
            \bfseries Gender Male, \emph n (\%) & 8740 (53.2)\\ 
            \bfseries Race, \emph n (\%) \\ 
                \hspace{10mm} White & 14896 (90.7) \\
                \hspace{10mm} Other/Not Recorded & 608 (3.7) \\
                \hspace{10mm} Black & 570 (3.5) \\
                \hspace{10mm} Hispanic & 170 (1.0) \\
                \hspace{10mm} Asian & 168 (1.0) \\
                \hspace{10mm} Indigenous & 16 (0.01) \\
            \bfseries \textit{APOE} Genotype, \emph n (\%) \\ 
                \hspace{10mm} \textit{APOE} ${\bm{\varepsilon}}$2 & 2028 (12.3) \\
                \hspace{10mm} \textit{APOE} ${\bm{\varepsilon}}$3 & 10177 (62.0) \\
                \hspace{10mm} \textit{APOE} ${\bm{\varepsilon}}$4 & 4223 (25.7) \\
            \bfseries Average Speciality Visits (SD) & 1.67 (4.6) \\ 
            \bfseries Average PCP Encounters (SD) &  5.25 (5.63) \\ 
            % \bfseriesEncounters AVG (SD)} & 14.55 (23.2) \\
            % \bfseriesKeyword Matches AVG (SD)} & 26 (54.03) \\
            \bottomrule
        \end{tabular}
        
}
{\caption{Demographics of Data}}

\end{table}

\begin{figure}[h!] 
\centering 
\includegraphics[scale = 0.35]{example-sequences.png}
\caption{Example Sequences and Always Patterns}
\end{figure}

\end{document}
