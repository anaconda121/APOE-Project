 % use the "wcp" class option for workshop and conference
 % proceedings
 %\documentclass[gray]{jmlr} % test grayscale version
 %\documentclass[tablecaption=bottom]{jmlr}% journal article
 \documentclass[pmlr,twocolumn,10pt]{jmlr} % W&CP article

% \usepackage{geometry}
% \geometry{margins=0.1in,textwidth=7in}

 % The following packages will be automatically loaded:
 % amsmath, amssymb, natbib, graphicx, url, algorithm2e

 %\usepackage{rotating}% for sideways figures and tables
 %\usepackage{longtable}% for long tables

 % The booktabs package is used by this sample document
 % (it provides \toprule, \midrule and \bottomrule).
 % Remove the next line if you don't require it.
\usepackage{tabto}    
\usepackage{graphicx}
\usepackage{natbib}
\usepackage{url}

\usepackage{booktabs}
 % The siunitx package is used by this sample document
 % to align numbers in a column by their decimal point.
 % Remove the next line if you don't require it.
\usepackage[load-configurations=version-1]{siunitx} % newer version 
%\usepackage{siunitx}

 % The following command is just for this sample document:
\newcommand{\cs}[1]{\texttt{\char`\\#1}}% remove this in your real article

% The following is to recognise equal contribution for authorship
\newcommand{\equal}[1]{{\hypersetup{linkcolor=black}\thanks{#1}}}

% Customized Tabs
\newcommand\mytab{\tab \hspace{-5cm}}

 % Define an unnumbered theorem just for this sample document for
 % illustrative purposes:
\theorembodyfont{\upshape}
\theoremheaderfont{\scshape}
\theorempostheader{:}
\theoremsep{\newline}
\newtheorem*{note}{Note}

 % change the arguments, as appropriate, in the following:
\jmlrvolume{LEAVE UNSET}
\jmlryear{2021}
\jmlrsubmitted{LEAVE UNSET}
\jmlrpublished{LEAVE UNSET}
\jmlrworkshop{Machine Learning for Health (ML4H) 2021} % W&CP title

 % The optional argument of \title is used in the header
\title[Short Title]{Full Title of Article\titlebreak This Title Has
A Line Break}

 % Anything in the title that should appear in the main title but 
 % not in the article's header or the volume's table of
 % contents should be placed inside \titletag{}

 %\title{Title of the Article\titletag{\thanks{Some footnote}}}


 % Use \Name{Author Name} to specify the name.
 % If the surname contains spaces, enclose the surname
 % in braces, e.g. \Name{John {Smith Jones}} similarly
 % if the name has a "von" part, e.g \Name{Jane {de Winter}}.
 % If the first letter in the forenames is a diacritic
 % enclose the diacritic in braces, e.g. \Name{{\'E}louise Smith}

 % \thanks must come after \Name{...} not inside the argument for
 % example \Name{John Smith}\nametag{\thanks{A note}} NOT \Name{John
 % Smith\thanks{A note}}

 % Anything in the name that should appear in the title but not in the 
 % article's header or footer or in the volume's
 % table of contents should be placed inside \nametag{}

 % Two authors with the same address
 % \author{%
 %  \Name{Author Name1\nametag{\thanks{A note}}} \Email{abc@sample.com}\and
 %  \Name{Author Name2} \Email{xyz@sample.com}\\
 %  \addr Address
 % }

 % Three or more authors with the same address:
 % \author{%
 %  \Name{Author Name1} \Email{an1@sample.com}\\
 %  \Name{Author Name2} \Email{an2@sample.com}\\
 %  \Name{Author Name3} \Email{an3@sample.com}\\
 %  \Name{Author Name4} \Email{an4@sample.com}\\
 %  \Name{Author Name5} \Email{an5@sample.com}\\
 %  \Name{Author Name6} \Email{an6@sample.com}\\
 %  \Name{Author Name7} \Email{an7@sample.com}\\
 %  \Name{Author Name8} \Email{an8@sample.com}\\
 %  \Name{Author Name9} \Email{an9@sample.com}\\
 %  \Name{Author Name10} \Email{an10@sample.com}\\
 %  \Name{Author Name11} \Email{an11@sample.com}\\
 %  \Name{Author Name12} \Email{an12@sample.com}\\
 %  \Name{Author Name13} \Email{an13@sample.com}\\
 %  \Name{Author Name14} \Email{an14@sample.com}\\
 %  \addr Address
 % }

 % Authors with different addresses and equal first authors:
\author{%
\Name{First Author 1}\equal{These authors contributed equally} \Email{abc@sample.com}\\
\addr University X, Country 1
\AND
% footnotemark[1] is to refer to the \equal footnote
\Name{First Author 2}\footnotemark[1] \Email{def@sample.com}\\
\addr University Y, Country 2
\AND
\Name{Last Author} \Email{ghi@sample.com}\\
\addr University Z, Country 3
}

\begin{document}

\maketitle

\begin{abstract}
\tab Dementia is a neurodegenerative disorder that causes cognitive decline and affects more than 50 million people worldwide. Dementia is under-diagnosed by healthcare professionals—only one in four people who suffer from dementia are diagnosed. Even when a diagnosis is made, it may not be entered as a structured diagnosis code in a patient’s charts.  Information relevant to cognitive impairment is often found within electronic health records but manual review of clinician notes by experts is both time consuming and often prone to errors. Automated evaluation of these notes presents an opportunity to label patients with cognitive impairment (CI) in real-world data. 
\end{abstract}
\begin{keywords}
EHR, NLP, Demetia
\end{keywords}

\section{Introduction}
\label{sec:intro}

This is a sample article that uses the \textsf{jmlr} class with
the \texttt{wcp} class option.  Please follow the guidelines in
this sample document as it can help to reduce complications when
combining the articles into a book. Please avoid using obsolete
commands, such as \verb|\rm|, and obsolete packages, such as
\textsf{epsfig}.\footnote{See
\url{http://www.ctan.org/pkg/l2tabu}} Some packages that are known
to cause problems for the production editing process are checked for
by the \textsf{jmlr} class and will generate an error. (If you want
to know more about the production editing process, have a look at
the video tutorials for the production editors at
\url{http://www.dickimaw-books.com/software/makejmlrbookgui/videos/}.)

Please also ensure that your document will compile with PDF\LaTeX.
If you have an error message that's puzzling you, first check for it
at the UK TUG FAQ
\url{https://texfaq.org/FAQ-man-latex}.  If
that doesn't help, create a minimal working example (see
\url{https://www.dickimaw-books.com/latex/minexample/}) and post
to somewhere like \TeX\ on StackExchange
(\url{http://tex.stackexchange.com/}) or the \LaTeX\ Community Forum
(\url{http://www.latex-community.org/forum/}).

\begin{note}
This is an numbered theorem-like environment that was defined in
this document's preamble.
\end{note}

\subsection{Sub-sections}

Sub-sections are produced using \verb|\subsection|.

\subsubsection{Sub-sub-sections}

Sub-sub-sections are produced using \verb|\subsubsection|.

\paragraph{Sub-sub-sub-sections}

Sub-sub-sub-sections are produced using \verb|\paragraph|.
These are unnumbered with a running head.

\subparagraph{Sub-sub-sub-sub-sections}

Sub-sub-sub-sub-sections are produced using \verb|\subparagraph|.
These are unnumbered with a running head.

\section{Related Works}

Always use \verb|\label| and \verb|\ref| (or one of the commands
described below) when cross-referencing.  For example, the next
section is Section~\ref{sec:math} but you can also refer to it using
\sectionref{sec:math}. The \textsf{jmlr} class
provides some convenient cross-referencing commands:
\verb|\sectionref|, \verb|\equationref|, \verb|\tableref|,
\verb|\figureref|, \verb|\algorithmref|, \verb|\theoremref|,
\verb|\lemmaref|, \verb|\remarkref|, \verb|\corollaryref|,
\verb|\definitionref|, \verb|\conjectureref|, \verb|\axiomref|,
\verb|\exampleref| and \verb|\appendixref|. The argument of these
commands may either be a single label or a comma-separated list
of labels. Examples:

Referencing sections: \sectionref{sec:math} or
\sectionref{sec:intro,sec:math} or
\sectionref{sec:intro,sec:math,sec:tables,sec:figures}.

Referencing equations: \equationref{eq:trigrule} or
\equationref{eq:trigrule,eq:df} or
\equationref{eq:trigrule,eq:f,eq:df,eq:y}.

Referencing tables: \tableref{tab:operatornames} or
\tableref{tab:operatornames,tab:example} or
\tableref{tab:operatornames,tab:example,tab:example-booktabs}.

Referencing figures: \figureref{fig:nodes} or
\figureref{fig:nodes,fig:teximage} or
\figureref{fig:nodes,fig:teximage,fig:subfigex} or
\figureref{fig:circle,fig:square}.

Referencing algorithms: \algorithmref{alg:gauss} or
\algorithmref{alg:gauss,alg:moore} or
\algorithmref{alg:gauss,alg:moore,alg:net}.

Referencing theorem-like environments: \theoremref{thm:eigenpow},
\lemmaref{lem:sample}, \remarkref{rem:sample}, 
\corollaryref{cor:sample}, \definitionref{def:sample},
\conjectureref{con:sample}, \axiomref{ax:sample} and
\exampleref{ex:sample}.

Referencing appendices: \appendixref{apd:first} or
\appendixref{apd:first,apd:second}.

\section{Dataset, Preprocessing, and Annotations}

\paragraph{Dataset}
\label{sec:Dataset} Our gold-standard dataset consisted of a cohort (N=16,428) of patients from the Mass General Brigham (MGB) HealthCare (formerly Partner's Healthcare, comprising two major academic hospitals, community hospitals, and community health centers in the Boston area) system who were older than 60 years (as of July 13, 2021) and had APOE genotype data available from the BioBank. Each patients' EHR record was then annotated by neurologists using a web-based annotation tool (UI Interface in Appendix A) as 1) Yes, i.e., patient has CI; 2) No i.e., Patient does not have CI; and 3) Neither i.e., sequence has no information on patient’s cognition.

\paragraph{Preprocessing}
\label{sec:Preprocessing} For each patient in our gold-standard dataset, we extracted unstructured clinician notes, identified matches to dementia-related keywords (listed in Table 1), and constructed 800 character sequences from the note text around each of these matches. Our cohort of 16,428 patients had 279,224 sequences in total. 

\begin{table*}[htbp]
\floatconts
  {tab:operatornames}%
  {\caption{Keywords indicative of Cognitive Impairment}}%
  {%
\begin{tabular}{lcc}
    \toprule
    \bfseries Keyword & \bfseries Match Count \\
    \midrule    

    \textbf{Memory} & \fseries 109218 \\ 
    \textbf{Cognition}  & \fseries 87655 \\ 
    \textbf{Dementia} & \fseries 51034 \\ 
    \textbf{Cerebral} & \fseries 45886 \\ 
    \textbf{Cerebrovascular} & \fseries 36370 \\ 
    \textbf{Cerebellar} & \fseries 26863 \\
    \textbf{Cognitive Impairment} & \fseries 20267 \\ 
    \textbf{Alzheimer} & \fseries 20581 \\ 
    \textbf{MOCA} & \fseries 9767 \\ 
    \textbf{Neurocognitive} & \fseries 7711 \\ 
    \textbf{MCI} & \fseries 3889 \\ 
    \textbf{Amnesia} & \fseries 3695 \\ 
    \textbf{AD} & \fseries 2673 \\ 
    \textbf{Lewy} & \fseries 2561 \\ 
    \textbf{MMSE} & \fseries 2134 \\ 
    \textbf{LBD} & \fseries 224 \\ 
    \textbf{Corticobasal} & \fseries 147 \\ 
    \textbf{Pick's} & \fseries 41 \\ 
    
    \bottomrule
\end{tabular}
}
\end{table*}

\paragraph{Annotations}
\label{sec:Annotations} We initially assigned 5,000 sequences diversified by keyword matches from 5,000 unique patients for neurologist labeling. As the manual annotation of 279,000+ sequences in not feasible, we devised a scheme to expedite annotations known as "always patterns". An always pattern is defined as a phrase or regex expression that in any context indicates the phrase will be labeled with a particular class (i.e. yes, no, or neither). If an always pattern is inputted, all other sequences have language that matches with the phrase will be automatically labeled accordingly. % {List any other sequence assignment schemes} %
Currently, 8,656 sequences have been annotated, 8,050 through always patterns and 606 manually. % {Will change} %

The final dataset was split between train (95\%) and holdout test (5\%) sets, stratified across label and proportion of sequences annotated manually and through always patterns. % {Will change} % 
Validation datsets were split from the train set using techniques described in the Methodology section. Table 2 shows demographics of the cohort of patients.  

\section{Methodology}
\label{sec:Methodology}  

We built x models and compared performance to baseline model. 

\paragraph{(1) Structured Features Model} 

\paragraph{(2) Logistic Regression with TF-IDF Vectors}

\paragraph{(3) Model 1 + Model 2}

\paragraph{(4) Transformer Based Sequence Classification Language Model}


\subsection{Algorithms}
\label{sec:algorithms}

Enumerated textual algorithms can be displayed using the
\texttt{algorithm} environment. Within this environment, use
\verb|\caption| to set the caption and you can use an
\texttt{enumerate} or nested \texttt{enumerate} environments.
For example, see \algorithmref{alg:gauss}. Note that algorithms
float like figures and tables.

\begin{algorithm}[htbp]
\floatconts
  {alg:gauss}%
  {\caption{The Gauss-Seidel Algorithm}}%
{%
\begin{enumerate}
  \item For $k=1$ to maximum number of iterations
    \begin{enumerate}
      \item For $i=1$ to $n$
        \begin{enumerate}
        \item $x_i^{(k)} = 
          \frac{b_i - \sum_{j=1}^{i-1}a_{ij}x_j^{(k)}
          - \sum_{j=i+1}^{n}a_{ij}x_j^{(k-1)}}{a_{ii}}$
        \item If $\|\vec{x}^{(k)}-\vec{x}^{(k-1)} < \epsilon\|$,
          where $\epsilon$ is a specified stopping criteria, stop.
      \end{enumerate}
    \end{enumerate}
\end{enumerate}
}%
\end{algorithm}

If you'd rather have the same numbering throughout the algorithm
but still want the convenient indentation of nested 
\texttt{enumerate} environments, you can use the
\texttt{enumerate*} environment provided by the \textsf{jmlr}
class. For example, see \algorithmref{alg:moore}.

\begin{algorithm}
\floatconts
  {alg:moore}%
  {\caption{Moore's Shortest Path}}%
{%
Given a connected graph $G$, where the length of each edge is 1:
\begin{enumerate*}
  \item Set the label of vertex $s$ to 0
  \item Set $i=0$
  \begin{enumerate*}
    \item \label{step:locate}Locate all unlabelled vertices 
          adjacent to a vertex labelled $i$ and label them $i+1$
    \item If vertex $t$ has been labelled,
    \begin{enumerate*}
      \item[] the shortest path can be found by backtracking, and 
      the length is given by the label of $t$.
    \end{enumerate*}
    otherwise
    \begin{enumerate*}
      \item[] increment $i$ and return to step~\ref{step:locate}
    \end{enumerate*}
  \end{enumerate*}
\end{enumerate*}
}%
\end{algorithm}

Pseudo code can be displayed using the \texttt{algorithm2e}
environment. This is defined by the \textsf{algorithm2e} package
(which is automatically loaded) so check the \textsf{algorithm2e}
documentation for further details.\footnote{Either \texttt{texdoc
algorithm2e} or \url{http://www.ctan.org/pkg/algorithm2e}}
For an example, see \algorithmref{alg:net}.

\begin{algorithm2e}
\caption{Computing Net Activation}
\label{alg:net}
 % older versions of algorithm2e have \dontprintsemicolon instead
 % of the following:
 %\DontPrintSemicolon
 % older versions of algorithm2e have \linesnumbered instead of the
 % following:
 %\LinesNumbered
\KwIn{$x_1, \ldots, x_n, w_1, \ldots, w_n$}
\KwOut{$y$, the net activation}
$y\leftarrow 0$\;
\For{$i\leftarrow 1$ \KwTo $n$}{
  $y \leftarrow y + w_i*x_i$\;
}
\end{algorithm2e}

\section{Description Lists}

The \textsf{jmlr} class also provides a description-like 
environment called \texttt{altdescription}. This has an
argument that should be the widest label in the list. Compare:
\begin{description}
\item[add] A method that adds two variables.
\item[differentiate] A method that differentiates a function.
\end{description}
with
\begin{altdescription}{differentiate}
\item[add] A method that adds two variables.
\item[differentiate] A method that differentiates a function.
\end{altdescription}

\section{Theorems, Lemmas etc}
\label{sec:theorems}

The following theorem-like environments are predefined by
the \textsf{jmlr} class: \texttt{theorem}, \texttt{example},
\texttt{lemma}, \texttt{proposition}, \texttt{remark}, 
\texttt{corollary}, \texttt{definition}, \texttt{conjecture}
and \texttt{axiom}. You can use the \texttt{proof} environment
to display the proof if need be, as in \theoremref{thm:eigenpow}.

\begin{theorem}[Eigenvalue Powers]\label{thm:eigenpow}
If $\lambda$ is an eigenvalue of $\vec{B}$ with eigenvector
$\vec{\xi}$, then $\lambda^n$ is an eigenvalue of $\vec{B}^n$
with eigenvector $\vec{\xi}$.
\begin{proof}
Let $\lambda$ be an eigenvalue of $\vec{B}$ with eigenvector
$\xi$, then
\begin{align*}
\vec{B}\vec{\xi} &= \lambda\vec{\xi}
\intertext{premultiply by $\vec{B}$:}
\vec{B}\vec{B}\vec{\xi} &= \vec{B}\lambda\vec{\xi}\\
\Rightarrow \vec{B}^2\vec{\xi} &= \lambda\vec{B}\vec{\xi}\\
&= \lambda\lambda\vec{\xi}\qquad
\text{since }\vec{B}\vec{\xi}=\lambda\vec{\xi}\\
&= \lambda^2\vec{\xi}
\end{align*}
Therefore true for $n=2$. Now assume true for $n=k$:
\begin{align*}
\vec{B}^k\vec{\xi} &= \lambda^k\vec{\xi}
\intertext{premultiply by $\vec{B}$:}
\vec{B}\vec{B}^k\vec{\xi} &= \vec{B}\lambda^k\vec{\xi}\\
\Rightarrow \vec{B}^{k+1}\vec{\xi} &= \lambda^k\vec{B}\vec{\xi}\\
&= \lambda^k\lambda\vec{\xi}\qquad
\text{since }\vec{B}\vec{\xi}=\lambda\vec{\xi}\\
&= \lambda^{k+1}\vec{\xi}
\end{align*}
Therefore true for $n=k+1$. Therefore, by induction, true for all
$n$.
\end{proof}
\end{theorem}

\begin{lemma}[A Sample Lemma]\label{lem:sample}
This is a lemma.
\end{lemma}

\begin{remark}[A Sample Remark]\label{rem:sample}
This is a remark.
\end{remark}

\begin{corollary}[A Sample Corollary]\label{cor:sample}
This is a corollary.
\end{corollary}

\begin{definition}[A Sample Definition]\label{def:sample}
This is a definition.
\end{definition}

\begin{conjecture}[A Sample Conjecture]\label{con:sample}
This is a conjecture.
\end{conjecture}

\begin{axiom}[A Sample Axiom]\label{ax:sample}
This is an axiom.
\end{axiom}

\begin{example}[An Example]\label{ex:sample}
This is an example.
\end{example}

\section{Color vs Grayscale}
\label{sec:color}

It's helpful if authors supply grayscale versions of their
images in the event that the article is to be incorporated into
a black and white printed book. With external PDF, PNG or JPG
graphic files, you just need to supply a grayscale version of the
file. For example, if the file is called \texttt{myimage.png},
then the gray version should be \texttt{myimage-gray.png} or
\texttt{myimage-gray.pdf} or \texttt{myimage-gray.jpg}. You don't
need to modify your code. The \textsf{jmlr} class checks for
the existence of the grayscale version if it is print mode 
(provided you have used \verb|\includegraphics| and haven't
specified the file extension).

You can use \verb|\ifprint| to determine which mode you are in.
For example, in \figureref{fig:nodes}, the 
\ifprint{dark gray}{purple} ellipse represents an input and the
\ifprint{light gray}{yellow} ellipse represents an output.
Another example: {\ifprint{\bfseries}{\color{red}}important text!}

You can use the class option \texttt{gray} to see how the
document will appear in gray scale mode. \textcolor{blue}{Colored
text} will automatically be converted to gray scale in print mode.

The \textsf{jmlr} class loads the \textsf{xcolor}
package, so you can also define your own colors. For example:
\ifprint
  {\definecolor{myred}{gray}{0.5}}%
  {\definecolor{myred}{rgb}{0.5,0,0}}%
\textcolor{myred}{XYZ}.

The \textsf{xcolor} class is loaded with the \texttt{x11names}
option, so you can use any of the x11 predefined colors (listed
in the \textsf{xcolor} documentation\footnote{either 
\texttt{texdoc xcolor} or \url{http://www.ctan.org/pkg/xcolor}}).

\section{Citations and Bibliography}
\label{sec:cite}

The \textsf{jmlr} class automatically loads \textsf{natbib}
and automatically sets the bibliography style, so you don't need to
use \verb|\bibliographystyle|.
This sample file has the citations defined in the accompanying
BibTeX file \texttt{jmlr-sample.bib}. For a parenthetical
citation use \verb|\citep|. For example
\citep{guyon-elisseeff-03}. For a textual citation use
\verb|\citet|. For example \citet{guyon2007causalreport}. 
Both commands may take a comma-separated list, for example
\citet{guyon-elisseeff-03,guyon2007causalreport}.

These commands have optional arguments and have a starred
version. See the \textsf{natbib} documentation for further
details.\footnote{Either \texttt{texdoc natbib} or
\url{http://www.ctan.org/pkg/natbib}}

The bibliography is displayed using \verb|\bibliography|.

\acks{Acknowledgements go here.}

\bibliography{jmlr-sample}

\appendix

\section{First Appendix}\label{apd:first}

This is the first appendix.

\section{Second Appendix}\label{apd:second}

This is the second appendix.

\end{document}
