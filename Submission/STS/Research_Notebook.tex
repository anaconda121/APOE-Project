\documentclass[11pt,letterpaper]{article}

\newcommand{\workingDate}{\textsc{\today}}
\newcommand{\userName}{Tanish Tyagi}
\newcommand{\institution}{Massachusetts General Hospital/Harvard Medical School}
\usepackage{researchdiary_png}
\usepackage{url}
\usepackage{amsmath}
\usepackage{natbib}
\bibliographystyle{abbrvnat}
\setcitestyle{authoryear,open={(},close={)}}
\usepackage{graphicx}
\graphicspath{ {images/} }

\usepackage{hyperref}
\hypersetup{
    colorlinks=true,
    linkcolor=blue,
    filecolor=blue,      
    urlcolor=blue,
    allcolors=blue,
    pdfpagemode=FullScreen,
    }
    
\begin{document}
\begin{titlepage}

\newcommand{\HRule}{\rule{\linewidth}{0.5mm}} % Defines a new command for the horizontal lines, change thickness here

\center
\includegraphics[scale=0.5]{images/mgh_logo.eps}\\[1cm] % Include a department/university logo - this will require the graphicx package
 
\center % Center everything on the page

\textsc{\LARGE NeuraHealth Research Notebook}\\[1.5cm] % Name of your university/college
\textsc{\Large Massachusetts General Hospital}\\[0.5cm] % Major heading such as course name
\textsc{\large MIND Data Science Lab}\\[0.5cm] % Minor heading such as course title

\begin{minipage}{0.4\textwidth}
\begin{flushleft} \large
\emph{Author:}\\
Tanish Tyagi % Your name
\end{flushleft}
\end{minipage}
~
\begin{minipage}{0.4\textwidth}
\begin{flushright} \large
\emph{Mentor:} \\
Dr. Sudeshna Das \\[1.2em] % Supervisor's Name
\end{flushright}
\end{minipage}\\[2cm]
\makeatother

{\large \today}\\[2cm]

\vfill

\end{titlepage}

\section{March 6, 2022}

Code Location: \texttt{APOE-V2/Analysis/Sequence v. Full Note Model Comparison.ipynb}

The main phenomenon I am currently investigating is why the model Xiao trained using full notes as opposed to sequences has completely different predictions than the sequence model I trained for ML4H. Today I made a distribution plot of the ratios of sequences to notes stratified by cognitive statuses. I used the annotations from NAT (\url{nat.partners.org}) for this experiment. 

\begin{figure}[ht!]
\centering
\includegraphics[width=10cm]{ratio_sequences_notes.png}
\caption{Distribution Plot of Ratio of Sequences to Notes by Cognitive Status}
\end{figure}

This figure confirmed my hypothesis that those with normal cognition would have a smaller sequence:note ratio. We also decided to have an average note:sequence ratio be around 0.1 based off this plot. 

\section{March 7, 2022}

Code Location: \texttt{APOE-V2/Analysis/Sequence v. Full Note Model Comparison.ipynb}

At the daily meeting today, we looked at the sequences that the full note model and sequence model were run on. We concluded that the sequence level model was more accurate in its predictions, and that even when the sequence text was equivalent to the entire note, both models would give completely different class predictions. We also noticed that the full note model seemed much more unsure in its predictions compared to the sequence level model, as shown by the figure below:

\begin{figure}[ht!]
\centering
\includegraphics[width=10cm]{predicted_probabilites_sequence_full_note.png}
\caption{Distribution Plot of Predicted Probabilites from Sequence and Full-Note models}
\end{figure}

Based of these observations, I will be meeting with Xiao tomorrow to go over her code for the full note model to see if there are any errors. 

\section{March 8, 2022}

I met with Xiao today to discuss the discrepancies between the sequence and full-note models. Upon looking at her code, I realized that she was using my sequence level model but instead making predictions using text of full-notes, meaning that there was no error in importing the model or with the attention weights as I had previously thought. In fact, all of her code was very similar to my prediction pipeline I made over the summer. Based off this, we both came to the conclusion that the process of using full-notes confuses the model as full-notes often contain conflicting information that can make a model uncertain and incorrect in its predictions.

\section{March 9, 2022}

Code Location: \texttt{APOE-V2/ClinicalBERT/ClinicalBERT NAT Sequence and Note Predictions.ipynb}

We noticed a minor bug in my training code that could affect the predicted class assignments and decided to retrain both the sequence and full-note models. The full-note model code also had some bugs, and we thought retraining it using my code would fix the low predicted probabilities and incorrect predictions to some extent. However, upon training, the probability distribution for both models remained the same and worse, the sequence level model predictions changed, leading the model to predict more sequences incorrectly. Will investigate more tomorrow.

\section{March 10 and 11, 2022}

My attention during these two days was primarily focused on getting annotations using the new design web app. The current clusters of sequences loaded in are matches from memory and cognition, with sub-clusters of having impair, having intact, having both intact and impair, or neither. I was able to annotate three chunks adding up 114 sequences. I have also marked 9 sequences for further consensus to bring up at a future meeting.

\section{March 14, 2022}

Code Location: \texttt{APOE-V2/ClinicalBERT/Analysis/ClinicalBERT Pipeline Inspection.ipynb}

In a further attempt to understand the differences between predicting on the sequence vs. full-note level, I extracted the attention weights for the model when a test sequence or test note is inputted for prediction. When the sequence is equal to the full-note are the exact same, the attention weights are the same as well. When the sequence is a snippet of the note, the attention weights are considerably different, which makes sense. 

\section{March 15, 2022}

Code Location: \texttt{APOE-V2/ClinicalBERT/Analysis/ClinicalBERT Pipeline Inspection.ipynb}

During the daily 11:30 meeting, I extracted the pooling layers for the models when it encounters a sequence vs. full-note. I got similar results to results I got when the attention weights were extracted, which once again made sense.

A minor breakthrough was made when we went back to further examining the predictions of the models that I trained on 3/9/2022. When the sequence equals the full note, both the predictions of the sequence model and full note model are the same as well as the probability, which actually makes a lot of sense since the attention weights and pooling layers are the same among others. 

\section{June 6, 2022}

Code Location: \texttt{APOE-V2/ClinicalBERT/Models/ClinicalBERT\_Training\_Script\_0606.ipynb}

Today was my first day doing full-time research at the lab since last summer, and it was a great feeling to finally get back into it! I used the ClinicalBERT training script I created to train the sequence level deep learning model and applied it to the current database of annotated sequences (around 4700). Future plans include to apply the ClinicalBERT model to a final dataset of around 6500 sequences so that it can become a great sequence level predictor. By improving the sequence level predictions, the logistic regression patient level predictor model will also become significantly improved. 

\section{June 7, 2022}

Code Location: \texttt{APOE-V2/ClinicalBERT/Models/ClinicalBERT\_Training\_Script\_0606.ipynb}

I encountered a weird error when running the optuna study that I had not seen before. The terminal said that said \texttt{CUDA error: device-side assert triggered. CUDA kernel errors might be asynchronously reported at some other API call,so the stacktrace below might be incorrect. For debugging consider passing CUDA\_LAUNCH\_BLOCKING=1.} This error really did not tell me much, so I asked my colleagues and they recommended checking the dataset to ensure there are no bugs and to run the script from the commmand line as a python file instead of a jupyter notebook while specifying the GPU ports to use.

None of the above fixes worked, but luckily I was able to find a fix which was to remove all the columns in my train, validation, and test sets except "text" and "labels" and restarting the kernel. This was quite a strange fix, but at the same time I was given a strange error. The model will finish training tomorrow. 

\section{June 8, 2022}

Code Location: \texttt{APOE-V2/ClinicalBERT/Models/ClinicalBERT\_Training\_Script\_0606.ipynb} \\ and \texttt{APOE-V2/ClinicalBERT/Analysis/ClinicalBERT\_Trial\_Analysis.ipynb}

The model finished training this morning. The best trial was trial 10: with 0.974 AUC and 0.897 ACC on the evalulation set. On the held out test set, the model achieved 0.829 ACC And 0.933 AUC. The hyperparameters that were varied throughout the study had the below values for trial 10:

\begin{itemize}
\item adam epilson: 5.3157E-06
\item early stopping patience: 3
\item learning rate: 5.5703E-05
\item num train epochs: 1
\end{itemize}

The AUC and ACC were not as good as the previous iteration of the model, which means that I will have to spend more time finding hyperparameters to tune. Some options for this include increasing the number of trials in the Optuna study from 20 to 40 (potentially), having the Optuna study maximize AUC instead of ACC, increasing the range of possible values for the learning rate, and finding other hyperparameters to include in the Optuna tuning process in addition to the four mentioned above. 

To dig deeper, I performed an analysis of the hyperparameters values for each trial and the corresponding AUC and ACC. 

\begin{figure}[ht!]
\label{auc_trial_20}
\centering
\includegraphics[width=10cm]{auc_trial.png}
\caption{AUC per Trial}
\end{figure}

\section{June 10, 2022}

To start the day, I cleaning up my folder on mindds and got my gitlab repo in order. I had neglected git duties for many months now, and I thought I would fix it today. Took a while, but I was able to get everything in order and resolve all merge conflicts. 

Now, I explored options with tuning the hyperparameters for ClinicalBERT. I found that the important hyperparameters were already being tuned via Optuna. Learning rate stood out to me as a very important parameter, and found the paper by \citep{sun2019fine} titled, "How to fine-tune BERT for text classification?" very helpful for this task. Tuning the learning rate is very important to prevent catastrophic forgetting \citep{mccloskey1989catastrophic},  where the pre-trained knowledge is erased during learning of new knowledge. In \citep{sun2019fine}, the authors trained BERT on text classification for movie reviews, which shares similarities with my task. They tested multiple learning rates, which had the results shown in the convergence graphs shown below:

\begin{figure}[ht!]
\centering
\includegraphics[width=17cm]{learning_rates.png}
\caption{Learning Rates and Convergence, Source: \citep{sun2019fine}}
\end{figure}

From the authors: ``We find that a lower learning rate, such as 2e-5, is necessary to make BERT overcome the catastrophic forgetting problem. With an aggressive learn rate of 4e-4, the training set fails to converge." With these results, I went to examine the learning rate of the best trial from my Optuna study. The learning rate was approximately 5e-5, and the loss on the evaluation set was 0.31. I figured that with an Optuna study of 40-50 trials, the model was come closer to a learning rate of 2e-5 and the loss would converge. 

I also closely examined my values for the number of training epochs. From online research, I found that the max number of epochs needed for NLP tasks and transformers in general is 5, so I increased the max number of possible training epochs to 5 instead of 3. 

I then started a 50 trial Optuna study to see if anything changes.

\section{June 11, 2022}
Code Location: \texttt{APOE-V2/ClinicalBERT/Models/ClinicalBERT\_Training\_Script\_0610.ipynb}

Having an Optuna trial run for 50 trials did not change the performance much, if anything at all. The best trial was \#6, which was fairly disappointing considering that this was quite early in the study. The hyperparameters for this trial are as follows:

\begin{itemize}
\item adam epilson: 2.4209E-06
\item early stopping patience: 2
\item learning rate: 6.6236E-06
\item num train epochs: 3
\end{itemize}

The learning did decrease from approximately 5E-5, but I am wondering whether this learning rate has the model learning too slow. I decided to also investigate the adam epsilon parameter and any way to tune the adam optimizer in general. After reading through numerous blog posts and scouring complicated equations I didn't understand, I came to the conclusion that the epilson parameter should be the only one that needs to be tuned during the training process.  

The adam optimizer combines the RMSProp and Momentum optimizers. The equations for the adam optimizer are as follows: 

\begin{center}
$m_0 = 0, v_0 = 0$ \\
Modified Momentum Equation: $m_{t+1} = \beta_{1}m_t + (1 - \beta_{1})\nabla_{\theta_{j}}\mathcal{L}(\theta)$ \\
Modified RMSProp Equation: $v_{t+1} = \beta_{2}m_t + (1 - \beta_{2})\nabla_{\theta_{j}}\mathcal{L}(\theta)^2$ \\
Correction: $\hat{m}_{t+1} = \frac{m_{t+1}}{1-(\beta_1)^2}$ and $\hat{v}_{t+1} = \frac{m_{t+1}}{1-(\beta_2)^2}$ \\
$\theta_{j+1} = \theta_{j} - \frac{\epsilon}{\sqrt{v_{t+1}} + 1e^{-5}}m_{t+1}$
\end{center}

The $\beta_1$ and $\beta_2$ parameters are repeatedly multiplied with each other, so they should start at numbers close to 1 so that over time they approach 0 as gradient descent comes to a close. Pytorch, along with all other popular deep learning libraries, initializes $\beta_1$ to 0.9 and $\beta_2$ to 0.999, so they are fine as is. The tensorflow documentation, however, cautions against leaving the adam epilson value as is, saying, ``The default value of 1e-8 for epsilon might not be a good default in general. For example, when training an Inception network on ImageNet a current good choice is 1.0 or 0.1." So my current tuning mechanisms seem to be holding up.

\section{June 13, 2022}

Code Locations: \texttt{APOE-V2/ClinicalBERT/Analysis/ClinicalBERT\_Trial\_Analysis.py} \\ and \texttt{APOE-V2/ClinicalBERT/Models/ClinicalBERT\_Sequence\_Model.py}

First, I created a general \texttt{Trial Analysis} class that I could use for gathering all the needed data from my Optuna studies and applied it to my 50 trial study.

\begin{figure}[h!]
\centering
\includegraphics[width=12cm]{50_trial.png}
\caption{AUC per trial for 50 trial study}
\end{figure}

This graph is similar to the previous graph, as there is no clear trend. Upon looking at the learning rates chosen by Optuna and their corresponding AUCs, all learning rates of form \texttt{x} $\cdot 10^{-5}$ yield the highest AUCs. After trials with this learning rates, the sampler would mysteriously recommend a learning rate much smaller, which resulted in poor performance. The trial analysis spreadsheet is full of such trends, and yet the sampler is unable to pick up on the pattern.  

On a separate note, I created a class for the \texttt{ClinicalBERT Sequence Model} for further hyperparameter tuning trials. I am running into a database error which I will address tomorrow.

\section{June 14, 2022}

Code Locations: \texttt{APOE-V2/ClinicalBERT/Models/ClinicalBERT\_Sequence\_Model.py} \\ and \texttt{APOE-V2/ClinicalBERT/Models/ClinicalBERT\_Training\_0616.ipynb}

I fixed the errors with the \texttt{ClinicalBERT Sequence Model} class. The gold standard dataset consists of $\approx$ 6500 annotated sequences with keywords and 500 sequences that contain no keywords. I started training the model with one edit: changing the range of the learning from 1E-6 to 1E-4 as learning rates under 1E-6 inhibited the model's learning. 

\section{June 15, 2022}

Code Location: \texttt{APOE-V2/ClinicalBERT/Models/ClinicalBERT\_Training\_0616.ipynb}

I realized that there was an error in the generation of the sequences that did not have a keyword match and had to stop training. The performance was quite similar to the previous study, but I believe there is potential for higher performance with correctly chosen sequences with no keywords.

\section{June 16, 2022} 

Code Location: \texttt{APOE-V2/ClinicalBERT/Models/ClinicalBERT\_Training\_0616.ipynb}

I trained on solely the 6500 sequences with a possible range of learning rates from 1E-6 to 1E-3 to further explore the learning rate space. I found the performance was extremely poor (0.5 AUC), as the model was learning too fast and forgetting past knowledge and patterns. 

\section{June 17, 2022} 

Code Location: \texttt{APOE-V2/ClinicalBERT/Models/ClinicalBERT\_Training\_0616.ipynb} \\ and \texttt{APOE-V2/ClinicalBERT/Preprocessing/Generating\_No\_Keyword\_Sequences.ipynb}

I created the new set of 500 sequences by sampling random 800 character portions from notes and started running a 50 study Optuna trial. The training will finish over the weekend, after which I will conduct a thorough analysis of the loss curves and parameter configurations.

\clearpage
\bibliography{main.bib}
\nocite{*}

\end{document}
